%\documentclass[11pt,reqno]{amsart}
\documentclass[11pt,reqno]{article}
\usepackage[margin=.8in, paperwidth=8.5in, paperheight=11in]{geometry}
%\usepackage{geometry}                % See geometry.pdf to learn the layout options. There are lots.
%\geometry{letterpaper}                   % ... or a4paper or a5paper or ... 
%\geometry{landscape}                % Activate for for rotated page geometry
%\usepackage[parfill]{parskip}    % Activate to begin paragraphs with an empty line rather than an indent7
\usepackage{graphicx}
\usepackage{pstricks}
\usepackage{amssymb}
\usepackage{epstopdf}
\usepackage{amsmath}
\usepackage{subfigure}
\usepackage{caption}
\pagestyle{plain}
%\renewcommand{\topfraction}{0.3}
%\renewcommand{\bottomfraction}{0.8}
%\renewcommand{\textfraction}{0.07}
\DeclareGraphicsRule{.tif}{png}{.png}{`convert #1 `dirname #1`/`basename #1 .tif`.png}

\title{Foundations of Computer Architecture: \\ Problem Set 3 }
\author{Andrew Rickert}
\date{Professor: Dr. Horace Malcolm \\ \hspace{-19pt} Due Date: February 21,  2012}                                           % Activate to display a given date or no date

\begin{document}
\maketitle


% Page 1
\begin{flushleft} 
Problem 1 \\
\rule{500pt}{1pt}\\
\end{flushleft} 
We can solve the problems below by converting the decimal values to their binary equivalents then converting them to the recoded versions. These values are:
\begin{eqnarray*}
539736088_{ten} &=& 00100000 00101011 10111000 00011000_{two} \\
-73_{ten} &=& 11111111 11111111 11111111 10110111_{two} \\ 
\end{eqnarray*}
These values can be recoded as 
\begin{eqnarray*}
539736088_{ten} =&+& 01000000 00000000 00000000 00000000 \\
&-&00100000 00000000 00000000 00000000 \\
&+&00000000 01000000 00000000 00000000 \\
&-&00000000 00100000 00000000 00000000 \\
&+&00000000 00010000 00000000 00000000 \\
&-&00000000 00001000 00000000 00000000 \\
&+&00000000 00000100 00000000 00000000 \\
&-&00000000 00000000 10000000 00000000 \\
&+&00000000 00000000 01000000 00000000 \\
&-&00000000 00000000 00001000 00000000 \\
&+&00000000 00000000 00000000 00100000 \\
&-&00000000 00000000 00000000 00001000 \\
-73_{ten} =&+&100000000 00000000 00000000 00000000 \\
&-&\hspace{6pt}00000000 00000000 00000000 10000000 \\
&+&\hspace{6pt}00000000 00000000 00000000 01000000 \\
&-&\hspace{6pt}00000000 00000000 00000000 00010000 \\
&+&\hspace{6pt}00000000 00000000 00000000 00001000 \\
&-&\hspace{6pt}00000000 00000000 00000000 00000001 \\
\end{eqnarray*}
\newpage

\noindent\framebox{Part a}\\ 
Looking at the recoded value of 539736088 we can see we would need 6 additions.\\

\noindent\framebox{Part b}\\ 
Looking at the recoded value of -73 we can see we would need 3 additions.\\

\noindent\framebox{Part c}\\ 
Looking at the recoded value of 539736088 we can see we would need 6 subtractions.\\

\noindent\framebox{Part d}\\ 
Looking at the recoded value of -73 we can see we would need 3 subtractions.\\


\begin{flushleft} 
Problem 2 \\
\rule{500pt}{1pt}\\
\end{flushleft} 

Looking at the MIPS floating point instruction encoding table on page 261 of we can see that with $\$f2 = 00010$ and $\$f6 = 00110$ the instruction will be encoded as \\
\begin{tabular}{| c | c | c | c | c | c |}
\hline
010001 & 10000 & 00000 & 00110 & 00010 & 100000\\
\hline
\end{tabular}\\

\noindent Now we convert to the hex representation
\[ 0100\hspace{6pt}0110\hspace{6pt}0000\hspace{6pt}0000\hspace{6pt}0011\hspace{6pt}0000\hspace{6pt}1010\hspace{6pt}0000_{two} = 0\text{x}460030\text{A}0 \]

\begin{flushleft} 
Problem 3 \\
\rule{500pt}{1pt}\\
\end{flushleft} 

\noindent The two's complement representation of the speed of light is 
\[ 186262_{ten} = 00000000 00000010 11010111 10010110_{two} \]
The gives yields the hex value directly
\[ \$8 = 0\text{x}0002\text{D}796\]

To find the floating point value first we need to convert the value to normalized form:
\begin{eqnarray*}
10 11010111 10010110.0 \times 2^0 = 1.0 11010111 10010110 2^{17}
\end{eqnarray*}
The biased exponent is $17 + 127 = 144 = 10010000_{two}$. The sign bit is 0 and the remaining 23 bits for the fraction can be read off from the normalized form above. This leads to the following encoded value\\ \\
\begin{tabular}{| c | c | c |}
\hline
 0 & 10010000 & 01101011110010110000000\\
\hline
\end{tabular}\\

\noindent We can now convert the value to a hex representation
\[
 01001000 00110101 11100101 10000000 = 0\text{x}4835\text{E}580
\]
\newpage

\begin{flushleft} 
Problem 4 \\
\rule{500pt}{1pt}\\
\end{flushleft} 
\framebox{Part a}\\ 

The opcode 001000 tells us that the instruction is an add immediate. We can place the remainder of the bits in the required I-format:\\ \\
\begin{tabular}{| c | c | c | c |}
\hline
001000 & 00001 & 01011 & 10111000 00011000\\
\hline
\end{tabular}\\

The first set of 5 bits after the opcode determine the source register. The value 00001 refers to the compiler reserved register $\$at$. The second set of 5 bits refers to the destination register; the value 01011 is register $\$t3$. So the full symbolic instruction is 
\[ \text{addi} \quad \$t3,\$at,0\text{x}\text{B}818 \]

\noindent\framebox{Part b}\\ 
The full binary for the instruction and immediate value are
\begin{eqnarray*}
0\text{x}202\text{BB}818 &=& 00100000001010111011100000011000_{two} \\
550039612_{ten} &=& 00100000 11001000 11110000 00111100_{two} \\
\end{eqnarray*}

Xoring th�se values gives the result that remains in register $\$23$.
\[ \$23 = 00000000 11100011 01001000 00100100 \]

The opcode is 000000 for this instruction so we need to look at the last 6 bits to determine the 'function'. The value 100100 determines this instruction to be an \emph{and} instruction. We have an R-format instruction as follows\\
\begin{tabular}{| c | c | c | c | c | c |}
\hline
000000 & 00111 & 00011 & 01001 & 00000 & 100100 \\
\hline
\end{tabular}\\

The first set of 5 bits is the source register; 00111 is register $\$a3$. The second 5 bits is the target register; 00011 is register $\$v1$. The last 5 bits are the destination register; 01001 is register $\$t0$. So the final form of the instruction is 
\[ \text{and} \quad \$t0,\$a3,\$v1 \]

\end{document}  